% !TEX root = main.tex

%%%%%%%%%%%%%%%%%%%%%%% required packages %%%%%%%%%%%%%%%%%%%%%%%%%%
% Packages required for our macros

% add headers (for draft mode): https://ctan.org/pkg/fancyhdr
\usepackage{fancyhdr}

% format URLs with \url{...}: https://ctan.org/pkg/url
\usepackage{url}

% format text in color: https://ctan.org/pkg/color
\usepackage{color}

% format dates: https://ctan.org/pkg/datetime
\usepackage[mmddyyyy,hhmmss]{datetime}

% remove excess spaces after macros: https://ctan.org/pkg/xspace
\usepackage{xspace}

% enhanced support for graphics: https://ctan.org/pkg/graphicx
\usepackage{graphicx}
\DeclareGraphicsExtensions{.pdf,.png,.eps,.ps,.jpg}

%%%%%%%%%%%%%%%%%%%%%%% optional packages %%%%%%%%%%%%%%%%%%%%%%%%%%
% Useful packages [optional]
% (If you add anything here, describe it, link to its CTAN page, and keep it commented out)

% compact numbered citations: https://ctan.org/pkg/cite
%\usepackage{cite}

% For professional tables: https://ctan.org/pkg/booktabs
%\usepackage{booktabs}

% AMS math commands: https://ctan.org/pkg/amsmath
%\usepackage{amsmath} 

% micro-typographic extensions: https://ctan.org/pkg/microtype
% keeps words from spilling outside column.
% \usepackage{microtype}

% writing numbers in SI units: https://ctan.org/pkg/siunitx
% \usepackage{siunitx}

% Scientific plotting: https://www.overleaf.com/learn/latex/Pgfplots_package
% \usepackage{pgfplots}
% \pgfplotsset{compat=1.15}

% Acronyms: https://ctan.org/pkg/acronym?lang=en
% \usepackage[nolist]{acronym}
% \begin{acronym}
%  \acro{SHORT}{Long name}
% \end{acronym}

%%%%%%%%%%%%%%%%%%%%%%%%%%%%%%%%%%%%%%%%%%%%%%%%%%
%%%%%%%%%%%%%%%%%%% MACROS %%%%%%%%%%%%%%%%%%%%%%%

% Simplify writing 'et al.' after an author name.
\newcommand{\etal}{et~al.\xspace}
% Simplify writing Wi-Fi and avoid line breaks.
\newcommand{\wifi}{\mbox{Wi-Fi}\xspace}

%%%%%%%%%%%%%%%%%%% comments %%%%%%%%%%%%%%%%%%%%%%%
% Macros for notes and comments - they disappear in submit mode.
\ifsubmit
  \newcommand{\hey}[1]{\relax}
  \newcommand{\heyvarun}[1]{\relax}
  \newcommand{\bibtex}[1]{\relax}
  \newcommand{\note}[1]{\relax}
\else
  % quick inline comment:
  \newcommand{\hey}[1]{\textcolor{magenta}{[{#1}]}}
  \newcommand{\heyvarun}[1]{\textcolor{blue}{[Varun: {#1}]}}
  % a paragraph comment:
  \newcommand{\note}[1]{\par\textcolor{magenta}{Note: {#1}}\par}
  % a reference that should go to bibtex:
  \newcommand{\bibtex}[1]{\textcolor{red}{@bibtex}\{#1\}} 
\fi

% Bug: If there is no empty line between two consecutive \note{} the second one appears as normal text.

% Anything in \hide{} is ignored.
\newcommand{\hide}[1]{\relax}


%%%%%%%%%%%%%%%%%%% figures %%%%%%%%%%%%%%%%%%%%%%%
% reference a figure
\newcommand{\figlabel}[1]{\label{fig:#1}}
\newcommand{\figref}[1]{Figure~\ref{fig:#1}}

% insert a narrow (1-column) figure, from a graphics file,
% using the filename for the ref key.
% size is a fraction of linewidth, typically 0.9 or 1.0
% \addfigure{filename without ext}{size}{caption}
\newcommand{\addfigure}[3]{
\begin{figure}[tbp]
\centerline{\resizebox{#2\linewidth}{!}{\includegraphics{figs/#1}}}
\caption{\figlabel{#1}#3}
\end{figure}
}

% insert a narrow (1-column) figure, from a latex file,
% using the filename for the ref key.
% size is a fraction of linewidth, typically 0.9 or 1.0
% \addfigure{filename without ext}{size}{caption}
\newcommand{\addfiguretex}[3]{
\begin{figure}[tbp]
\centerline{\resizebox{#2\linewidth}{!}{\input{figs/#1}}}
\caption{\figlabel{#1}#3}
\end{figure}
}


%%%%%%%%%%%%%%%%%%% tables %%%%%%%%%%%%%%%%%%%%%%%

% reference a table
\newcommand{\tablabel}[1]{\label{tab:#1}}
\newcommand{\tabref}[1]{Table~\ref{tab:#1}}

% insert a narrow (1-column) table, using the filename for the ref key.
% size is a fraction of linewidth, typically 0.9 or 1.0
% \addtable{filename without ext}{size}{caption}
\newcommand{\addtable}[3]{
\begin{table}[tbp]
\caption{\tablabel{#1}#3}
\centerline{\resizebox{#2\linewidth}{!}{\input{tabs/#1}}}
\end{table}
}

%%%%%%%%%%%%%%%%%%% sections %%%%%%%%%%%%%%%%%%%%%%%

% reference a section
\newcommand{\seclabel}[1]{\label{sec:#1}}
\newcommand{\secref}[1]{Section~\ref{sec:#1}}

%%%%%%%%%%%%%%%%%%%%%%%%%%%%%%%%%%%%%%%%%%%%%%%%%%%%%
