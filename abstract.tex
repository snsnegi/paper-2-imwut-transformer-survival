% !TEX root = main.tex

% ABSTRACT - edit this file to write your abstract.

\begin{abstract}

% Just-In-Time Adaptive Interventions (JITAIs) aim to support health behavior by providing the right support at the right time. A critical determinant of JITAI efficacy is timing the delivery such that the user is `receptive', defined as the cognitive and behavioral capacity to receive, process, and use the provided support.. While prior research has explored context sensing to predict receptivity, standard approaches typically operationalize this construct as a binary outcome within a fixed window. However, theoretical definitions suggest that human availability is better characterized as a continuous, time-varying state. Transitioning to such fine-grained modeling significantly increases the complexity of the learning task; while deep sequence models offer the capacity to capture these dynamics from high-dimensional sensor streams, their effectiveness in mHealth is often constrained by the scarcity of labeled interaction data.

% To address this gap, we propose \texttt{PRISM}, a deep learning framework designed to handle the sparsity and heterogeneity of longitudinal mobile sensing. The architecture employs a Channel-Independent Transformer (PatchTST) encoder coupled with a discrete-time survival objective, modeling receptivity as a probabilistic time-to-event distribution. Crucially, to mitigate the labeled data bottleneck, we employ a self-supervised learning strategy: we pre-train the encoder on a large corpus of unlabeled sensor traces, allowing the model to learn fundamental behavioral patterns via Masked Patch Reconstruction before fine-tuning.

% We evaluated this approach on the \textit{Lvl UP} intervention dataset, utilizing unlabeled data from preliminary feasibility studies for pre-training and the main efficacy trial for evaluation. Our results demonstrate that the \texttt{XYZAB} architecture achieves predictive performance competitive with established receptivity detection benchmarks while offering significant advantages in deployment scalability. By leveraging self-supervised initialization, the model effectively mitigates the "cold-start" problem and the heavy personalization burden inherent in deep learning applications, reaching stable performance ($AUC > 0.65$) within three training epochs. Furthermore, our approach demonstrates superior stability across challenging cross-validation folds compared to standard supervised baselines, effectively bridging distribution shifts between diverse user subsets. These findings suggest that \texttt{XYZAB} provides a robust, data-efficient pathway for deploying time-aware models that remain resilient in real-world mHealth scenarios.



% Varun's shorter example:
Just-In-Time Adaptive Interventions (JITAIs) aim to support health behavior by providing the right support at the right time. A critical determinant of JITAI efficacy is timing delivery such that the user is \textit{receptive}, defined as the cognitive and behavioral capacity to receive, process, and use support. While prior work has explored context sensing to predict receptivity, standard approaches typically operationalize this construct as a binary outcome within a fixed window, despite theoretical definitions characterizing availability as a continuous, time-varying state. Modeling receptivity at this granularity increases learning complexity, and deep sequence models are further constrained by the scarcity of labeled interaction data in mHealth settings.

To address these challenges, we propose \texttt{PRISM}, a deep learning framework for modeling receptivity as a probabilistic time-to-event distribution from longitudinal mobile sensing data. PRISM combines a Channel-Independent Transformer (PatchTST) encoder with a discrete-time survival objective and employs self-supervised pre-training on unlabeled sensor traces via Masked Patch Reconstruction to mitigate label scarcity.

We evaluate PRISM on the \textit{Lvl UP} intervention dataset, leveraging data from preliminary studies for pre-training and a large-scale efficacy trial for evaluation. Our results demonstrate competitive performance with established receptivity benchmarks. Self-supervised initialization yields up to 15.5\% improvement in median AUC across heterogeneous user groups, effectively mitigating the cold-start problem. These findings suggest that PRISM provides a data-efficient pathway for deploying resilient, time-aware receptivity models in real-world mHealth systems.
\end{abstract}

